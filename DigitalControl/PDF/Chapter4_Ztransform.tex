\documentclass[11pt]{article}

    \usepackage[breakable]{tcolorbox}
    \usepackage{parskip} % Stop auto-indenting (to mimic markdown behaviour)
    
    \usepackage{iftex}
    \ifPDFTeX
    	\usepackage[T1]{fontenc}
    	\usepackage{mathpazo}
    \else
    	\usepackage{fontspec}
    \fi

    % Basic figure setup, for now with no caption control since it's done
    % automatically by Pandoc (which extracts ![](path) syntax from Markdown).
    \usepackage{graphicx}
    % Maintain compatibility with old templates. Remove in nbconvert 6.0
    \let\Oldincludegraphics\includegraphics
    % Ensure that by default, figures have no caption (until we provide a
    % proper Figure object with a Caption API and a way to capture that
    % in the conversion process - todo).
    \usepackage{caption}
    \DeclareCaptionFormat{nocaption}{}
    \captionsetup{format=nocaption,aboveskip=0pt,belowskip=0pt}

    \usepackage[Export]{adjustbox} % Used to constrain images to a maximum size
    \adjustboxset{max size={0.9\linewidth}{0.9\paperheight}}
    \usepackage{float}
    \floatplacement{figure}{H} % forces figures to be placed at the correct location
    \usepackage{xcolor} % Allow colors to be defined
    \usepackage{enumerate} % Needed for markdown enumerations to work
    \usepackage{geometry} % Used to adjust the document margins
    \usepackage{amsmath} % Equations
    \usepackage{amssymb} % Equations
    \usepackage{textcomp} % defines textquotesingle
    % Hack from http://tex.stackexchange.com/a/47451/13684:
    \AtBeginDocument{%
        \def\PYZsq{\textquotesingle}% Upright quotes in Pygmentized code
    }
    \usepackage{upquote} % Upright quotes for verbatim code
    \usepackage{eurosym} % defines \euro
    \usepackage[mathletters]{ucs} % Extended unicode (utf-8) support
    \usepackage{fancyvrb} % verbatim replacement that allows latex
    \usepackage{grffile} % extends the file name processing of package graphics 
                         % to support a larger range
    \makeatletter % fix for grffile with XeLaTeX
    \def\Gread@@xetex#1{%
      \IfFileExists{"\Gin@base".bb}%
      {\Gread@eps{\Gin@base.bb}}%
      {\Gread@@xetex@aux#1}%
    }
    \makeatother

    % The hyperref package gives us a pdf with properly built
    % internal navigation ('pdf bookmarks' for the table of contents,
    % internal cross-reference links, web links for URLs, etc.)
    \usepackage{hyperref}
    % The default LaTeX title has an obnoxious amount of whitespace. By default,
    % titling removes some of it. It also provides customization options.
    \usepackage{titling}
    \usepackage{longtable} % longtable support required by pandoc >1.10
    \usepackage{booktabs}  % table support for pandoc > 1.12.2
    \usepackage[inline]{enumitem} % IRkernel/repr support (it uses the enumerate* environment)
    \usepackage[normalem]{ulem} % ulem is needed to support strikethroughs (\sout)
                                % normalem makes italics be italics, not underlines
    \usepackage{mathrsfs}
    

    
    % Colors for the hyperref package
    \definecolor{urlcolor}{rgb}{0,.145,.698}
    \definecolor{linkcolor}{rgb}{.71,0.21,0.01}
    \definecolor{citecolor}{rgb}{.12,.54,.11}

    % ANSI colors
    \definecolor{ansi-black}{HTML}{3E424D}
    \definecolor{ansi-black-intense}{HTML}{282C36}
    \definecolor{ansi-red}{HTML}{E75C58}
    \definecolor{ansi-red-intense}{HTML}{B22B31}
    \definecolor{ansi-green}{HTML}{00A250}
    \definecolor{ansi-green-intense}{HTML}{007427}
    \definecolor{ansi-yellow}{HTML}{DDB62B}
    \definecolor{ansi-yellow-intense}{HTML}{B27D12}
    \definecolor{ansi-blue}{HTML}{208FFB}
    \definecolor{ansi-blue-intense}{HTML}{0065CA}
    \definecolor{ansi-magenta}{HTML}{D160C4}
    \definecolor{ansi-magenta-intense}{HTML}{A03196}
    \definecolor{ansi-cyan}{HTML}{60C6C8}
    \definecolor{ansi-cyan-intense}{HTML}{258F8F}
    \definecolor{ansi-white}{HTML}{C5C1B4}
    \definecolor{ansi-white-intense}{HTML}{A1A6B2}
    \definecolor{ansi-default-inverse-fg}{HTML}{FFFFFF}
    \definecolor{ansi-default-inverse-bg}{HTML}{000000}

    % commands and environments needed by pandoc snippets
    % extracted from the output of `pandoc -s`
    \providecommand{\tightlist}{%
      \setlength{\itemsep}{0pt}\setlength{\parskip}{0pt}}
    \DefineVerbatimEnvironment{Highlighting}{Verbatim}{commandchars=\\\{\}}
    % Add ',fontsize=\small' for more characters per line
    \newenvironment{Shaded}{}{}
    \newcommand{\KeywordTok}[1]{\textcolor[rgb]{0.00,0.44,0.13}{\textbf{{#1}}}}
    \newcommand{\DataTypeTok}[1]{\textcolor[rgb]{0.56,0.13,0.00}{{#1}}}
    \newcommand{\DecValTok}[1]{\textcolor[rgb]{0.25,0.63,0.44}{{#1}}}
    \newcommand{\BaseNTok}[1]{\textcolor[rgb]{0.25,0.63,0.44}{{#1}}}
    \newcommand{\FloatTok}[1]{\textcolor[rgb]{0.25,0.63,0.44}{{#1}}}
    \newcommand{\CharTok}[1]{\textcolor[rgb]{0.25,0.44,0.63}{{#1}}}
    \newcommand{\StringTok}[1]{\textcolor[rgb]{0.25,0.44,0.63}{{#1}}}
    \newcommand{\CommentTok}[1]{\textcolor[rgb]{0.38,0.63,0.69}{\textit{{#1}}}}
    \newcommand{\OtherTok}[1]{\textcolor[rgb]{0.00,0.44,0.13}{{#1}}}
    \newcommand{\AlertTok}[1]{\textcolor[rgb]{1.00,0.00,0.00}{\textbf{{#1}}}}
    \newcommand{\FunctionTok}[1]{\textcolor[rgb]{0.02,0.16,0.49}{{#1}}}
    \newcommand{\RegionMarkerTok}[1]{{#1}}
    \newcommand{\ErrorTok}[1]{\textcolor[rgb]{1.00,0.00,0.00}{\textbf{{#1}}}}
    \newcommand{\NormalTok}[1]{{#1}}
    
    % Additional commands for more recent versions of Pandoc
    \newcommand{\ConstantTok}[1]{\textcolor[rgb]{0.53,0.00,0.00}{{#1}}}
    \newcommand{\SpecialCharTok}[1]{\textcolor[rgb]{0.25,0.44,0.63}{{#1}}}
    \newcommand{\VerbatimStringTok}[1]{\textcolor[rgb]{0.25,0.44,0.63}{{#1}}}
    \newcommand{\SpecialStringTok}[1]{\textcolor[rgb]{0.73,0.40,0.53}{{#1}}}
    \newcommand{\ImportTok}[1]{{#1}}
    \newcommand{\DocumentationTok}[1]{\textcolor[rgb]{0.73,0.13,0.13}{\textit{{#1}}}}
    \newcommand{\AnnotationTok}[1]{\textcolor[rgb]{0.38,0.63,0.69}{\textbf{\textit{{#1}}}}}
    \newcommand{\CommentVarTok}[1]{\textcolor[rgb]{0.38,0.63,0.69}{\textbf{\textit{{#1}}}}}
    \newcommand{\VariableTok}[1]{\textcolor[rgb]{0.10,0.09,0.49}{{#1}}}
    \newcommand{\ControlFlowTok}[1]{\textcolor[rgb]{0.00,0.44,0.13}{\textbf{{#1}}}}
    \newcommand{\OperatorTok}[1]{\textcolor[rgb]{0.40,0.40,0.40}{{#1}}}
    \newcommand{\BuiltInTok}[1]{{#1}}
    \newcommand{\ExtensionTok}[1]{{#1}}
    \newcommand{\PreprocessorTok}[1]{\textcolor[rgb]{0.74,0.48,0.00}{{#1}}}
    \newcommand{\AttributeTok}[1]{\textcolor[rgb]{0.49,0.56,0.16}{{#1}}}
    \newcommand{\InformationTok}[1]{\textcolor[rgb]{0.38,0.63,0.69}{\textbf{\textit{{#1}}}}}
    \newcommand{\WarningTok}[1]{\textcolor[rgb]{0.38,0.63,0.69}{\textbf{\textit{{#1}}}}}
    
    
    % Define a nice break command that doesn't care if a line doesn't already
    % exist.
    \def\br{\hspace*{\fill} \\* }
    % Math Jax compatibility definitions
    \def\gt{>}
    \def\lt{<}
    \let\Oldtex\TeX
    \let\Oldlatex\LaTeX
    \renewcommand{\TeX}{\textrm{\Oldtex}}
    \renewcommand{\LaTeX}{\textrm{\Oldlatex}}
    % Document parameters
    % Document title
    \title{Chapter4\_Ztransform}
    
    
    
    
    
% Pygments definitions
\makeatletter
\def\PY@reset{\let\PY@it=\relax \let\PY@bf=\relax%
    \let\PY@ul=\relax \let\PY@tc=\relax%
    \let\PY@bc=\relax \let\PY@ff=\relax}
\def\PY@tok#1{\csname PY@tok@#1\endcsname}
\def\PY@toks#1+{\ifx\relax#1\empty\else%
    \PY@tok{#1}\expandafter\PY@toks\fi}
\def\PY@do#1{\PY@bc{\PY@tc{\PY@ul{%
    \PY@it{\PY@bf{\PY@ff{#1}}}}}}}
\def\PY#1#2{\PY@reset\PY@toks#1+\relax+\PY@do{#2}}

\expandafter\def\csname PY@tok@gu\endcsname{\let\PY@bf=\textbf\def\PY@tc##1{\textcolor[rgb]{0.50,0.00,0.50}{##1}}}
\expandafter\def\csname PY@tok@s1\endcsname{\def\PY@tc##1{\textcolor[rgb]{0.73,0.13,0.13}{##1}}}
\expandafter\def\csname PY@tok@k\endcsname{\let\PY@bf=\textbf\def\PY@tc##1{\textcolor[rgb]{0.00,0.50,0.00}{##1}}}
\expandafter\def\csname PY@tok@mh\endcsname{\def\PY@tc##1{\textcolor[rgb]{0.40,0.40,0.40}{##1}}}
\expandafter\def\csname PY@tok@nt\endcsname{\let\PY@bf=\textbf\def\PY@tc##1{\textcolor[rgb]{0.00,0.50,0.00}{##1}}}
\expandafter\def\csname PY@tok@gr\endcsname{\def\PY@tc##1{\textcolor[rgb]{1.00,0.00,0.00}{##1}}}
\expandafter\def\csname PY@tok@nf\endcsname{\def\PY@tc##1{\textcolor[rgb]{0.00,0.00,1.00}{##1}}}
\expandafter\def\csname PY@tok@gi\endcsname{\def\PY@tc##1{\textcolor[rgb]{0.00,0.63,0.00}{##1}}}
\expandafter\def\csname PY@tok@nb\endcsname{\def\PY@tc##1{\textcolor[rgb]{0.00,0.50,0.00}{##1}}}
\expandafter\def\csname PY@tok@se\endcsname{\let\PY@bf=\textbf\def\PY@tc##1{\textcolor[rgb]{0.73,0.40,0.13}{##1}}}
\expandafter\def\csname PY@tok@sa\endcsname{\def\PY@tc##1{\textcolor[rgb]{0.73,0.13,0.13}{##1}}}
\expandafter\def\csname PY@tok@cpf\endcsname{\let\PY@it=\textit\def\PY@tc##1{\textcolor[rgb]{0.25,0.50,0.50}{##1}}}
\expandafter\def\csname PY@tok@gh\endcsname{\let\PY@bf=\textbf\def\PY@tc##1{\textcolor[rgb]{0.00,0.00,0.50}{##1}}}
\expandafter\def\csname PY@tok@kr\endcsname{\let\PY@bf=\textbf\def\PY@tc##1{\textcolor[rgb]{0.00,0.50,0.00}{##1}}}
\expandafter\def\csname PY@tok@ge\endcsname{\let\PY@it=\textit}
\expandafter\def\csname PY@tok@il\endcsname{\def\PY@tc##1{\textcolor[rgb]{0.40,0.40,0.40}{##1}}}
\expandafter\def\csname PY@tok@kp\endcsname{\def\PY@tc##1{\textcolor[rgb]{0.00,0.50,0.00}{##1}}}
\expandafter\def\csname PY@tok@fm\endcsname{\def\PY@tc##1{\textcolor[rgb]{0.00,0.00,1.00}{##1}}}
\expandafter\def\csname PY@tok@err\endcsname{\def\PY@bc##1{\setlength{\fboxsep}{0pt}\fcolorbox[rgb]{1.00,0.00,0.00}{1,1,1}{\strut ##1}}}
\expandafter\def\csname PY@tok@ss\endcsname{\def\PY@tc##1{\textcolor[rgb]{0.10,0.09,0.49}{##1}}}
\expandafter\def\csname PY@tok@kn\endcsname{\let\PY@bf=\textbf\def\PY@tc##1{\textcolor[rgb]{0.00,0.50,0.00}{##1}}}
\expandafter\def\csname PY@tok@mb\endcsname{\def\PY@tc##1{\textcolor[rgb]{0.40,0.40,0.40}{##1}}}
\expandafter\def\csname PY@tok@mo\endcsname{\def\PY@tc##1{\textcolor[rgb]{0.40,0.40,0.40}{##1}}}
\expandafter\def\csname PY@tok@vg\endcsname{\def\PY@tc##1{\textcolor[rgb]{0.10,0.09,0.49}{##1}}}
\expandafter\def\csname PY@tok@cs\endcsname{\let\PY@it=\textit\def\PY@tc##1{\textcolor[rgb]{0.25,0.50,0.50}{##1}}}
\expandafter\def\csname PY@tok@cm\endcsname{\let\PY@it=\textit\def\PY@tc##1{\textcolor[rgb]{0.25,0.50,0.50}{##1}}}
\expandafter\def\csname PY@tok@na\endcsname{\def\PY@tc##1{\textcolor[rgb]{0.49,0.56,0.16}{##1}}}
\expandafter\def\csname PY@tok@dl\endcsname{\def\PY@tc##1{\textcolor[rgb]{0.73,0.13,0.13}{##1}}}
\expandafter\def\csname PY@tok@vm\endcsname{\def\PY@tc##1{\textcolor[rgb]{0.10,0.09,0.49}{##1}}}
\expandafter\def\csname PY@tok@cp\endcsname{\def\PY@tc##1{\textcolor[rgb]{0.74,0.48,0.00}{##1}}}
\expandafter\def\csname PY@tok@vc\endcsname{\def\PY@tc##1{\textcolor[rgb]{0.10,0.09,0.49}{##1}}}
\expandafter\def\csname PY@tok@gp\endcsname{\let\PY@bf=\textbf\def\PY@tc##1{\textcolor[rgb]{0.00,0.00,0.50}{##1}}}
\expandafter\def\csname PY@tok@sr\endcsname{\def\PY@tc##1{\textcolor[rgb]{0.73,0.40,0.53}{##1}}}
\expandafter\def\csname PY@tok@nd\endcsname{\def\PY@tc##1{\textcolor[rgb]{0.67,0.13,1.00}{##1}}}
\expandafter\def\csname PY@tok@sc\endcsname{\def\PY@tc##1{\textcolor[rgb]{0.73,0.13,0.13}{##1}}}
\expandafter\def\csname PY@tok@kd\endcsname{\let\PY@bf=\textbf\def\PY@tc##1{\textcolor[rgb]{0.00,0.50,0.00}{##1}}}
\expandafter\def\csname PY@tok@c1\endcsname{\let\PY@it=\textit\def\PY@tc##1{\textcolor[rgb]{0.25,0.50,0.50}{##1}}}
\expandafter\def\csname PY@tok@ch\endcsname{\let\PY@it=\textit\def\PY@tc##1{\textcolor[rgb]{0.25,0.50,0.50}{##1}}}
\expandafter\def\csname PY@tok@w\endcsname{\def\PY@tc##1{\textcolor[rgb]{0.73,0.73,0.73}{##1}}}
\expandafter\def\csname PY@tok@kc\endcsname{\let\PY@bf=\textbf\def\PY@tc##1{\textcolor[rgb]{0.00,0.50,0.00}{##1}}}
\expandafter\def\csname PY@tok@sb\endcsname{\def\PY@tc##1{\textcolor[rgb]{0.73,0.13,0.13}{##1}}}
\expandafter\def\csname PY@tok@ow\endcsname{\let\PY@bf=\textbf\def\PY@tc##1{\textcolor[rgb]{0.67,0.13,1.00}{##1}}}
\expandafter\def\csname PY@tok@nv\endcsname{\def\PY@tc##1{\textcolor[rgb]{0.10,0.09,0.49}{##1}}}
\expandafter\def\csname PY@tok@sx\endcsname{\def\PY@tc##1{\textcolor[rgb]{0.00,0.50,0.00}{##1}}}
\expandafter\def\csname PY@tok@gs\endcsname{\let\PY@bf=\textbf}
\expandafter\def\csname PY@tok@vi\endcsname{\def\PY@tc##1{\textcolor[rgb]{0.10,0.09,0.49}{##1}}}
\expandafter\def\csname PY@tok@ne\endcsname{\let\PY@bf=\textbf\def\PY@tc##1{\textcolor[rgb]{0.82,0.25,0.23}{##1}}}
\expandafter\def\csname PY@tok@sh\endcsname{\def\PY@tc##1{\textcolor[rgb]{0.73,0.13,0.13}{##1}}}
\expandafter\def\csname PY@tok@o\endcsname{\def\PY@tc##1{\textcolor[rgb]{0.40,0.40,0.40}{##1}}}
\expandafter\def\csname PY@tok@gd\endcsname{\def\PY@tc##1{\textcolor[rgb]{0.63,0.00,0.00}{##1}}}
\expandafter\def\csname PY@tok@nc\endcsname{\let\PY@bf=\textbf\def\PY@tc##1{\textcolor[rgb]{0.00,0.00,1.00}{##1}}}
\expandafter\def\csname PY@tok@s2\endcsname{\def\PY@tc##1{\textcolor[rgb]{0.73,0.13,0.13}{##1}}}
\expandafter\def\csname PY@tok@gt\endcsname{\def\PY@tc##1{\textcolor[rgb]{0.00,0.27,0.87}{##1}}}
\expandafter\def\csname PY@tok@c\endcsname{\let\PY@it=\textit\def\PY@tc##1{\textcolor[rgb]{0.25,0.50,0.50}{##1}}}
\expandafter\def\csname PY@tok@m\endcsname{\def\PY@tc##1{\textcolor[rgb]{0.40,0.40,0.40}{##1}}}
\expandafter\def\csname PY@tok@kt\endcsname{\def\PY@tc##1{\textcolor[rgb]{0.69,0.00,0.25}{##1}}}
\expandafter\def\csname PY@tok@nl\endcsname{\def\PY@tc##1{\textcolor[rgb]{0.63,0.63,0.00}{##1}}}
\expandafter\def\csname PY@tok@nn\endcsname{\let\PY@bf=\textbf\def\PY@tc##1{\textcolor[rgb]{0.00,0.00,1.00}{##1}}}
\expandafter\def\csname PY@tok@mf\endcsname{\def\PY@tc##1{\textcolor[rgb]{0.40,0.40,0.40}{##1}}}
\expandafter\def\csname PY@tok@ni\endcsname{\let\PY@bf=\textbf\def\PY@tc##1{\textcolor[rgb]{0.60,0.60,0.60}{##1}}}
\expandafter\def\csname PY@tok@no\endcsname{\def\PY@tc##1{\textcolor[rgb]{0.53,0.00,0.00}{##1}}}
\expandafter\def\csname PY@tok@s\endcsname{\def\PY@tc##1{\textcolor[rgb]{0.73,0.13,0.13}{##1}}}
\expandafter\def\csname PY@tok@go\endcsname{\def\PY@tc##1{\textcolor[rgb]{0.53,0.53,0.53}{##1}}}
\expandafter\def\csname PY@tok@bp\endcsname{\def\PY@tc##1{\textcolor[rgb]{0.00,0.50,0.00}{##1}}}
\expandafter\def\csname PY@tok@sd\endcsname{\let\PY@it=\textit\def\PY@tc##1{\textcolor[rgb]{0.73,0.13,0.13}{##1}}}
\expandafter\def\csname PY@tok@mi\endcsname{\def\PY@tc##1{\textcolor[rgb]{0.40,0.40,0.40}{##1}}}
\expandafter\def\csname PY@tok@si\endcsname{\let\PY@bf=\textbf\def\PY@tc##1{\textcolor[rgb]{0.73,0.40,0.53}{##1}}}

\def\PYZbs{\char`\\}
\def\PYZus{\char`\_}
\def\PYZob{\char`\{}
\def\PYZcb{\char`\}}
\def\PYZca{\char`\^}
\def\PYZam{\char`\&}
\def\PYZlt{\char`\<}
\def\PYZgt{\char`\>}
\def\PYZsh{\char`\#}
\def\PYZpc{\char`\%}
\def\PYZdl{\char`\$}
\def\PYZhy{\char`\-}
\def\PYZsq{\char`\'}
\def\PYZdq{\char`\"}
\def\PYZti{\char`\~}
% for compatibility with earlier versions
\def\PYZat{@}
\def\PYZlb{[}
\def\PYZrb{]}
\makeatother


    % For linebreaks inside Verbatim environment from package fancyvrb. 
    \makeatletter
        \newbox\Wrappedcontinuationbox 
        \newbox\Wrappedvisiblespacebox 
        \newcommand*\Wrappedvisiblespace {\textcolor{red}{\textvisiblespace}} 
        \newcommand*\Wrappedcontinuationsymbol {\textcolor{red}{\llap{\tiny$\m@th\hookrightarrow$}}} 
        \newcommand*\Wrappedcontinuationindent {3ex } 
        \newcommand*\Wrappedafterbreak {\kern\Wrappedcontinuationindent\copy\Wrappedcontinuationbox} 
        % Take advantage of the already applied Pygments mark-up to insert 
        % potential linebreaks for TeX processing. 
        %        {, <, #, %, $, ' and ": go to next line. 
        %        _, }, ^, &, >, - and ~: stay at end of broken line. 
        % Use of \textquotesingle for straight quote. 
        \newcommand*\Wrappedbreaksatspecials {% 
            \def\PYGZus{\discretionary{\char`\_}{\Wrappedafterbreak}{\char`\_}}% 
            \def\PYGZob{\discretionary{}{\Wrappedafterbreak\char`\{}{\char`\{}}% 
            \def\PYGZcb{\discretionary{\char`\}}{\Wrappedafterbreak}{\char`\}}}% 
            \def\PYGZca{\discretionary{\char`\^}{\Wrappedafterbreak}{\char`\^}}% 
            \def\PYGZam{\discretionary{\char`\&}{\Wrappedafterbreak}{\char`\&}}% 
            \def\PYGZlt{\discretionary{}{\Wrappedafterbreak\char`\<}{\char`\<}}% 
            \def\PYGZgt{\discretionary{\char`\>}{\Wrappedafterbreak}{\char`\>}}% 
            \def\PYGZsh{\discretionary{}{\Wrappedafterbreak\char`\#}{\char`\#}}% 
            \def\PYGZpc{\discretionary{}{\Wrappedafterbreak\char`\%}{\char`\%}}% 
            \def\PYGZdl{\discretionary{}{\Wrappedafterbreak\char`\$}{\char`\$}}% 
            \def\PYGZhy{\discretionary{\char`\-}{\Wrappedafterbreak}{\char`\-}}% 
            \def\PYGZsq{\discretionary{}{\Wrappedafterbreak\textquotesingle}{\textquotesingle}}% 
            \def\PYGZdq{\discretionary{}{\Wrappedafterbreak\char`\"}{\char`\"}}% 
            \def\PYGZti{\discretionary{\char`\~}{\Wrappedafterbreak}{\char`\~}}% 
        } 
        % Some characters . , ; ? ! / are not pygmentized. 
        % This macro makes them "active" and they will insert potential linebreaks 
        \newcommand*\Wrappedbreaksatpunct {% 
            \lccode`\~`\.\lowercase{\def~}{\discretionary{\hbox{\char`\.}}{\Wrappedafterbreak}{\hbox{\char`\.}}}% 
            \lccode`\~`\,\lowercase{\def~}{\discretionary{\hbox{\char`\,}}{\Wrappedafterbreak}{\hbox{\char`\,}}}% 
            \lccode`\~`\;\lowercase{\def~}{\discretionary{\hbox{\char`\;}}{\Wrappedafterbreak}{\hbox{\char`\;}}}% 
            \lccode`\~`\:\lowercase{\def~}{\discretionary{\hbox{\char`\:}}{\Wrappedafterbreak}{\hbox{\char`\:}}}% 
            \lccode`\~`\?\lowercase{\def~}{\discretionary{\hbox{\char`\?}}{\Wrappedafterbreak}{\hbox{\char`\?}}}% 
            \lccode`\~`\!\lowercase{\def~}{\discretionary{\hbox{\char`\!}}{\Wrappedafterbreak}{\hbox{\char`\!}}}% 
            \lccode`\~`\/\lowercase{\def~}{\discretionary{\hbox{\char`\/}}{\Wrappedafterbreak}{\hbox{\char`\/}}}% 
            \catcode`\.\active
            \catcode`\,\active 
            \catcode`\;\active
            \catcode`\:\active
            \catcode`\?\active
            \catcode`\!\active
            \catcode`\/\active 
            \lccode`\~`\~ 	
        }
    \makeatother

    \let\OriginalVerbatim=\Verbatim
    \makeatletter
    \renewcommand{\Verbatim}[1][1]{%
        %\parskip\z@skip
        \sbox\Wrappedcontinuationbox {\Wrappedcontinuationsymbol}%
        \sbox\Wrappedvisiblespacebox {\FV@SetupFont\Wrappedvisiblespace}%
        \def\FancyVerbFormatLine ##1{\hsize\linewidth
            \vtop{\raggedright\hyphenpenalty\z@\exhyphenpenalty\z@
                \doublehyphendemerits\z@\finalhyphendemerits\z@
                \strut ##1\strut}%
        }%
        % If the linebreak is at a space, the latter will be displayed as visible
        % space at end of first line, and a continuation symbol starts next line.
        % Stretch/shrink are however usually zero for typewriter font.
        \def\FV@Space {%
            \nobreak\hskip\z@ plus\fontdimen3\font minus\fontdimen4\font
            \discretionary{\copy\Wrappedvisiblespacebox}{\Wrappedafterbreak}
            {\kern\fontdimen2\font}%
        }%
        
        % Allow breaks at special characters using \PYG... macros.
        \Wrappedbreaksatspecials
        % Breaks at punctuation characters . , ; ? ! and / need catcode=\active 	
        \OriginalVerbatim[#1,codes*=\Wrappedbreaksatpunct]%
    }
    \makeatother

    % Exact colors from NB
    \definecolor{incolor}{HTML}{303F9F}
    \definecolor{outcolor}{HTML}{D84315}
    \definecolor{cellborder}{HTML}{CFCFCF}
    \definecolor{cellbackground}{HTML}{F7F7F7}
    
    % prompt
    \makeatletter
    \newcommand{\boxspacing}{\kern\kvtcb@left@rule\kern\kvtcb@boxsep}
    \makeatother
    \newcommand{\prompt}[4]{
        \ttfamily\llap{{\color{#2}[#3]:\hspace{3pt}#4}}\vspace{-\baselineskip}
    }
    

    
    % Prevent overflowing lines due to hard-to-break entities
    \sloppy 
    % Setup hyperref package
    \hypersetup{
      breaklinks=true,  % so long urls are correctly broken across lines
      colorlinks=true,
      urlcolor=urlcolor,
      linkcolor=linkcolor,
      citecolor=citecolor,
      }
    % Slightly bigger margins than the latex defaults
    
    \geometry{verbose,tmargin=1in,bmargin=1in,lmargin=1in,rmargin=1in}
    
    

\begin{document}
    
    \maketitle
    
    

    
    \section{4. Transformées en Z}\label{transformuxe9es-en-z}

    \subsection{4.1 Définitions}\label{duxe9finitions}

    \subsubsection{Définition 4.1}\label{duxe9finition-4.1}

La transformée en z d'un signal discret \(\big\{w(kh)\big\}\), dénotée
\(W(z)\) ou \(\mathscr{Z}\left\{w(kh)\right\}\), est définie par la
série de puissances négatices suivante, où z est une variable complexe:

\[ W(z) = \mathscr{Z}\big\{w(kh)\big\} = \sum_{k=0}^{\infty}w(kh)z^{-k} \]

Cette série converge pour \(\|z\|>r\).

    \subsubsection{~Définition 4.2}\label{duxe9finition-4.2}

La transformée en z inverse d'une fonction
\(W:\mathbb{C} \rightarrow \mathbb{C}\) est le signal discret
\(\big\{w(kh)\big\}\), dénoté \(\mathscr{Z}^{-1}\big(W(z)\big)\), tel
que \(\mathscr{Z}\big\{w(kh)\big\}=W(z)\)

    \paragraph{Example}\label{example}

Le signal discret \(\big\{w(kh)\big\}\) de durée finie est défini ainsi:
\[ \big\{w(kh)\big\} = \{ 0, \mathbf{1}, 3, -2, 0, 0 \}\] L'élément en
gras représente la valeur du signal pour \(k=0\).

Pour \(z\neq0\), la transformée de ce signal est donnée par:
\[ \mathscr{Z}\{w(kh)\} = 1 + 3z^{-1} -2z^{-2} \]

    \subsection{4.2 Propriétés de la transformée en
Z}\label{propriuxe9tuxe9s-de-la-transformuxe9e-en-z}

    Les propriétés de la transformée en Z sont résumées ci-après:

\paragraph{Linéarité}\label{linuxe9arituxe9}

\[ \mathscr{Z}\left(\left\{w_1(kh)\right\}+\left\{w_2(kh)\right\}\right) = \mathscr{Z}\left\{w_1(kh)\right\}+\mathscr{Z}\left\{w_2(kh)\right\} \]
\[ \mathscr{Z}\left(a\left\{w(kh)\right\}\right) = a\mathscr{Z}\left\{w(kh)\right\} \]

\paragraph{Décalages temporels}\label{duxe9calages-temporels}

\[ \mathscr{Z}\left\{w(kh-dh)\right\} = z^{-d}W(z) \]
\[ \mathscr{Z}\left\{w(kh+dh)\right\} = z^{d}W(z) - \sum_{i=0}^{d-1}w(ih)z^{d-i} \]

\paragraph{Dérivation complexe}\label{duxe9rivation-complexe}

\[ \mathscr{Z}\left\{khw(kh)\right\} = -hz\frac{dW}{dz}(z) \]

\paragraph{Changement d'échelle
complexe}\label{changement-duxe9chelle-complexe}

\[ \mathscr{Z}\left\{a^{kh}w(kh)\right\} = W\left(\frac{z}{a^h}\right) \]

\paragraph{Valeurs initiales et
finale}\label{valeurs-initiales-et-finale}

\[ w(0) = \lim_{z \rightarrow \infty}W(z) \]
\[ \lim_{k \rightarrow \infty} w(kh) = \lim_{z \rightarrow 1}(z-1)W(z) \]

\paragraph{Produit de convolution}\label{produit-de-convolution}

\[ \mathscr{Z}\left\{\sum_{l=0}^{k}u(lh)g(kh-lh)\right\} = G(z)U(z) \]

\paragraph{Accumulation}\label{accumulation}

\[ \mathscr{Z}\left\{\sum_{l=0}^{k}w(lh)\right\} = \frac{z}{z-1}W(z) \]

\paragraph{Différence}\label{diffuxe9rence}

\[ \mathscr{Z}\left\{w(kh)-w(kh-h)\right\} = \frac{z-1}{z}W(z) \]

\newpage
    \subsection{4.3 Table des transformées de Laplace et
Z}\label{table-des-transformuxe9es-de-laplace-et-z}

     \begin{longtable}[]{@{}lllll@{}}
\toprule
$\text{N°}$ & $w(t)$ & $\mathscr{L}\big(w(t)\big)$ & $w(kh)$ & $\mathscr{Z}\big\{w(kh)\big\}$ \\
\midrule
\endhead
$\text{1}$ & $\delta(t)$ & $1$ & & \\
$\text{2}$ & & & $\Delta(kh)$ & $1$ \\
$\text{3}$ & $1$ & $\frac{1}{s}$ & $1$ & $\frac{z}{z-1}$ \\
$\text{4}$ & $t$ & $\frac{1}{s^2}$ & $kh$ & $\frac{hz}{(z-1)^2}$ \\
$\text{5}$ & $\frac{1}{2}t^2$ & $\frac{1}{s^3}$ & $\frac{1}{2}(kh)^2$ & $\frac{h^2z(z+1)}{2(z-1)^3}$ \\
$\text{6}$ & $\frac{1}{(l-1)!}t^{l-1}$ & $\frac{1}{s^l}$ & $\frac{1}{(l-1)!}(kh)^{l-1}$ & $\lim_{a \rightarrow 0}\frac{(-1)^{l-1}}{(l-1)!}\frac{\partial^{l-1}}{\partial{}a^{l-1}}\left( \frac{z}{z-e^{-ah}} \right)$ \\
$\text{7}$ & $e^{-at}$ & $\frac{1}{s+a}$ & $e^{-akh}$ & $\frac{z}{z-e^{-ah}}$ \\
$\text{8}$ & $te^{-at}$ & $\frac{1}{(s+a)^2}$ & $khe^{-akh}$ & $\frac{he^{-ah}z}{(z-e^{-ah})^2}$ \\
$\text{9}$ & $\frac{1}{2}t^2e^{-at}$ & $\frac{1}{(s+a)^3}$ & $\frac{1}{2}(kh)^2e^{-akh}$ & $\frac{h^2e^{-ah}z(z-e^{-ah}+2e^{-ah})}{2(z-e^{-ah})^3}$ \\
$\text{10}$ & $\frac{1}{(l-1)!}t^{l-1}e^{-at}$ & $\frac{1}{(s+a)^l}$ & $\frac{1}{(l-1)!}(kh)^{l-1}e^{-akh}$ & $\frac{(-1)^{l-1}}{(l-1)!}\frac{\partial^{l-1}}{\partial{}a^{l-1}}\left( \frac{z}{z-e^{-ah}} \right)$ \\
$\text{11}$ & $\sin(\omega{}t)$ & $\frac{\omega}{s^2+\omega^2}$ & $\sin(\omega{}kh)$ & $\frac{\sin(\omega{}h)z}{z^2-2\cos(\omega{}h)z+1}$ \\
$\text{12}$ & $\cos(\omega{}t)$ & $\frac{s}{s^2+\omega^2}$ & $\cos(\omega{}kh)$ & $\frac{z(z-\cos(\omega{}h))}{z^2-2\cos(\omega{}h)z+1}$ \\
$\text{13}$ & $e^{-at}\sin(\omega{}t)$ & $\frac{\omega}{(s+a)^2+\omega^2}$ & $e^{-akh}\sin(\omega{}kh)$ & $\frac{e^{-ah}\sin(\omega{}h)z}{z^2-2e^{-ah}\cos(\omega{}h)z+e^{-2ah}}$ \\
$\text{14}$ & $e^{-at}\cos(\omega{}t)$ & $\frac{s+a}{(s+a)^2+\omega^2}$ & $e^{-akh}\cos(\omega{}kh)$ & $\frac{z(z-e^{-ah}\cos(\omega{}h))}{z^2-2e^{-ah}\cos(\omega{}h)z+e^{-2ah}}$ \\
$\text{15}$ & & & $a^k$ & $\frac{z}{z-a}$ \\
$\text{16}$ & & & $ka^{k-1}$ & $\frac{z}{(z-a)^2}$ \\
$\text{17}$ & & & $\frac{1}{2}k(k-1)a^{k-2}$ & $\frac{z}{(z-a)^3}$ \\
$\text{18}$ & & & $\frac{1}{(l-1)!}\left(\prod_{i=0}^{l-2}(k-i)\right)a^{k-l+1}$ & $\frac{z}{(z-a)^l}$ \\
\bottomrule
\end{longtable}

    \subsection{4.4 Calcul de la transformée en Z
inverse}\label{calcul-de-la-transformuxe9e-en-z-inverse}

    Soit la transformée en z rationnelle propre définie par:

\[ W(z) = \frac{b_0z^m+b_1z^{m-1}+\dots+b_m}{z^n+a_1z^{n-1}+\dots+a_n} \qquad n \geq m \]

Le dénominateur est dit \emph{monique}, c'est-à-dire que le coefficient
du terme d'ordre le plus élevé est \(1\).

    Il existe 3 méthodes de calcul de la transformée en Z inverse:

\begin{itemize}
\tightlist
\item
  Décomposition en somme de fractions simples
\item
  Intégration dans le plan complexe
\item
  Inversion numérique
\end{itemize}

L'intégration dans le plan complexe étant rarement utilisée en pratique,
elle ne sera pas montrée ici.

    \subsubsection{Décomposition en sommes de fractions
simples}\label{duxe9composition-en-sommes-de-fractions-simples}

    \paragraph{Exemple}\label{exemple}

\[ W(z) = \frac{3}{(z+1)(z-2)} \]

Afin de faire apparaître un z au numérateur et ainsi obtenir une forme
semblable au tableau des transformées, on cherche la décomposition en
somme de fractions simples de \(W(z)/z\).

\[ \frac{W(z)}{z} = \frac{3}{z(z+1)(z-2)} \]

La décomposition donne:

\[ \frac{W(z)}{z} = -\frac{1,5}{z}+\frac{1}{z+1}+\frac{0,5}{z-2} \]

En multipliant par z de chaque côté, on obtient:

\[ W(z) = -1,5+\frac{z}{z+1}+\frac{0,5z}{z-2} \]

D'où, la transformée inverse donne:

\[ w(kh) = -1,5\Delta(kh)+(-1)^k+0,5.2^k \] ***

    \subsubsection{Inversion numérique}\label{inversion-numuxe9rique}

    \paragraph{Théorème}\label{thuxe9oruxe8me}

La transformée en z inverse de W(z) est le signal discret fourni par la
récurrence suivante, dans laquelle \(a_i = b_i = 0\) lorsque
\(i \ge n\):

\[ w(kh) = b_k-\sum_{l=0}^{k-1}w(lh)a_{k-l} \qquad k \geq 0 \]

    \paragraph{Exemple}\label{exemple}

\[ W(z) = \frac{z^3-2z^2+2z}{z^3-4z^2+5z-2} \]

En appliquant le théorème, on obtient successivement:

\begin{align}
w(0) &= 1 \\
w(h) &= -2-1(-4) = 2 \\
w(2h) &= 2-\big(1 \cdot 5 + 2(-4)\big) = 5 \\
w(3h) &= 0-\big(1(-2) + 2 \cdot 5 + 5(-4)\big) = 12 \\
w(4h) &= 0-\big(1 \cdot 0 + 2(-2) + 5 \cdot 5 + 12(-4)\big) = 27 \\
\vdots
\end{align}

Soit:

\[ \left\{ w(kh) \right\} = \{ \dots, 0, \mathbf{1}, 2, 5, 12, 27, \dots \} \]
***

    Afin d'obtenir le même résultat manuellement, il suffit d'effectuer la
division polynômiale du numérateur par le dénominateur de \(W(z)\).

    \subsubsection{Importance du cercle
unité}\label{importance-du-cercle-unituxe9}

    Le bout de code suivant permet de vérifier aisément l'effet d'un pôle
sur le système.

    \begin{tcolorbox}[breakable, size=fbox, boxrule=1pt, pad at break*=1mm,colback=cellbackground, colframe=cellborder]
\prompt{In}{incolor}{1}{\boxspacing}
\begin{Verbatim}[commandchars=\\\{\}]
\PY{o}{\PYZpc{}}\PY{k}{matplotlib} inline
\PY{k+kn}{import} \PY{n+nn}{matplotlib}\PY{n+nn}{.}\PY{n+nn}{pyplot} \PY{k}{as} \PY{n+nn}{plt}
\PY{n}{plt}\PY{o}{.}\PY{n}{style}\PY{o}{.}\PY{n}{use}\PY{p}{(}\PY{l+s+s1}{\PYZsq{}}\PY{l+s+s1}{../my\PYZus{}params.mplstyle}\PY{l+s+s1}{\PYZsq{}}\PY{p}{)}
\end{Verbatim}
\end{tcolorbox}

    \begin{tcolorbox}[breakable, size=fbox, boxrule=1pt, pad at break*=1mm,colback=cellbackground, colframe=cellborder]
\prompt{In}{incolor}{2}{\boxspacing}
\begin{Verbatim}[commandchars=\\\{\}]
\PY{k+kn}{import} \PY{n+nn}{control}

\PY{n}{G} \PY{o}{=} \PY{n}{control}\PY{o}{.}\PY{n}{tf}\PY{p}{(}\PY{p}{[}\PY{l+m+mi}{1}\PY{p}{,} \PY{l+m+mi}{0}\PY{p}{]}\PY{p}{,} \PY{p}{[}\PY{l+m+mi}{1}\PY{p}{,} \PY{o}{\PYZhy{}}\PY{l+m+mi}{1}\PY{p}{]}\PY{p}{,} \PY{k+kc}{True}\PY{p}{)}

\PY{n}{control}\PY{o}{.}\PY{n}{pzmap}\PY{p}{(}\PY{n}{G}\PY{p}{)}
\PY{n}{t}\PY{p}{,} \PY{n}{y} \PY{o}{=} \PY{n}{control}\PY{o}{.}\PY{n}{impulse\PYZus{}response}\PY{p}{(}\PY{n}{G}\PY{p}{,} \PY{n}{T}\PY{o}{=}\PY{p}{[}\PY{n}{i}\PY{o}{/}\PY{l+m+mi}{50} \PY{k}{for} \PY{n}{i} \PY{o+ow}{in} \PY{n+nb}{range}\PY{p}{(}\PY{l+m+mi}{10}\PY{p}{)}\PY{p}{]}\PY{p}{)}                                                        

\PY{n}{fig}\PY{p}{,} \PY{n}{ax} \PY{o}{=} \PY{n}{plt}\PY{o}{.}\PY{n}{subplots}\PY{p}{(}\PY{p}{)}
\PY{n}{ax}\PY{o}{.}\PY{n}{plot}\PY{p}{(}\PY{n}{t}\PY{p}{,} \PY{n}{y}\PY{p}{,} \PY{l+s+s1}{\PYZsq{}}\PY{l+s+s1}{.}\PY{l+s+s1}{\PYZsq{}}\PY{p}{,} \PY{n}{label}\PY{o}{=}\PY{l+s+s1}{\PYZsq{}}\PY{l+s+s1}{y}\PY{l+s+s1}{\PYZsq{}}\PY{p}{)}
\PY{n}{ax}\PY{o}{.}\PY{n}{set\PYZus{}xlabel}\PY{p}{(}\PY{l+s+s1}{\PYZsq{}}\PY{l+s+s1}{time [s]}\PY{l+s+s1}{\PYZsq{}}\PY{p}{)}
\PY{n}{ax}\PY{o}{.}\PY{n}{set\PYZus{}ylabel}\PY{p}{(}\PY{l+s+s1}{\PYZsq{}}\PY{l+s+s1}{output}\PY{l+s+s1}{\PYZsq{}}\PY{p}{)}
\PY{n}{\PYZus{}} \PY{o}{=} \PY{n}{ax}\PY{o}{.}\PY{n}{legend}\PY{p}{(}\PY{p}{)}
\end{Verbatim}
\end{tcolorbox}

    \begin{center}
    \adjustimage{max size={0.9\linewidth}{0.9\paperheight}}{Chapter4_Ztransform_files/Chapter4_Ztransform_21_0.png}
    \end{center}
    { \hspace*{\fill} \\}
    
    \begin{center}
    \adjustimage{max size={0.9\linewidth}{0.9\paperheight}}{Chapter4_Ztransform_files/Chapter4_Ztransform_21_1.png}
    \end{center}
    { \hspace*{\fill} \\}
    
    \subsection{4.5 Fonction de transfert}\label{fonction-de-transfert}

    La sortie d'un système discret est calculée, comme en continu, par le
produit de convolution:

\[ y(kh) = \sum_{l=0}^{k}u(lh)g(kh-lh) \]

avec \(y, u, g\), la sortie, l'entrée et la réponse impulsionnelle du
système respectivement.

    Par la transformée vue précédemment, on obtient:

\[ Y(z) = G(z)U(z) \]

La transformée \(G(z)\) de la réponse impulsionnelle est appelée
\textbf{fonction de transfert discrète} du système.

    Cette expression permet de:

\begin{itemize}
\tightlist
\item
  calculer la sortie \(Y(z)\), lorsque \(G(z)\) et \(U(z)\) sont connus
\item
  estimer la fonction de transfert \(G(z)\), lorsque \(Y(z)\) et
  \(U(z)\)
\end{itemize}

    La fonction de transfert discrète permet, comme la fonction de transfert
continue, de transformer la relation \emph{entrée-sortie} en relation
algébrique. Les règles, déjà bien connues, d'algèbre de schémas blocs,
sont donc identiques aux règles algébriques régissant les schémas blocs
en continu. Elles ne seront donc pas rappelées ici.

    Lorsque le système est décri par une équation aux différences, la
fonction de transfert est calculée à partir du quotient \(Y(z)/U(z)\):

\[ y(k) + a_1 y(k-1) + \dots + a_n y(k-n) = b_0 u(k-d) + b_1 u(k-d-1) + \dots + b_m u(k-d-m) \]

Par la transformée en Z, on obtient:

\begin{align}
Y(z) + a_1 z^{-1}Y(z) + \dots + a_n z^{-n}Y(z) &= b_0 z^{-d}U(z) + b_1 z^{-d-1}U(z) + \dots + b_m z^{-d-m}U(z) \\
\left(1 + a_1 z^{-1} + \dots + a_n z^{-n}\right)Y(z) &= \left(b_0 z^{-d} + b_1 z^{-d-1} + \dots + b_m z^{-d-m}\right)U(z) \\
\frac{Y(z)}{U(z)} &= z^{-d}\frac{b_0+b_1z^{-1}+\dots+b_mz^{-m}}{1+a_1z^{-1}+\dots+a_nz^{-n}} \\
\frac{Y(z)}{U(z)} &= \frac{b_0z^m+b_1z^{m-1}+\dots+b_m}{z^n+a_1z^{n-1}+\dots+a_n}
\end{align}

La dernière ligne est obtenue en multipliant le numérateur et le
dénominateur par \(z^n\) et en remplaçant \(d\) par \(n-m\).

    Comme pour les fonctions de transfert continues, les définitions
suivantes sont d'application:

\begin{itemize}
\tightlist
\item
  les racines du numérateur sont appelées les pôles du système
\item
  les racines du dénominateur sont appelées les pôles du système
\item
  le dénominateur est appelé polynôme caractéristique
\item
  lorsque qu'un zéro possède un module plus grand que 1, le système est
  dit à non-minimum de phase
\item
  l'ordre du système est défini par le nombre \(n\) de ses pôles
\item
  la différence entre le nombre de pôles et de zéros \(d = n - m\) est
  appelé surplus de pôles
\end{itemize}

    En général, on privilégie la fonction de transfert en fonction de
puissances positives de \(z\) pour l'analyse du système, et de
puissances négatives pour les aspects temps réels.

    \paragraph{Exemple 1}\label{exemple-1}

L'intégrateur numérique est défini par l'équation aux différences
suivante:

\[ y(k+1) = y(k) + u(k)h \]

D'où sa fonction de transfert discrète:

\[ G(z) = \frac{h}{z-1} \]

Il possède un pôle et pas de zéro et est d'ordre 1.

Le pôle pôle \(z=1\) est caractéristique d'un effet intégrateur
numérique, de la même manière que le pôle \(s=0\) l'est pour
l'intégrateur analogique. ***

    \paragraph{Exemple 2}\label{exemple-2}

Le régulateur PI numérique est décrit par l'équation:

\[ u(k) - u(k-1) = K_p e(k) + K_p \left( \frac{h}{T_i} - 1 \right) e(k-1) \]

Sa fonction de transfert est:

\[ K(z) = K_p \frac{1 + \left( \frac{h}{T_i} - 1 \right) z^{-1}}{1 - z^{-1}} \]

Ou encore:

\[ K(z) = K_p \frac{z + \frac{h}{T_i} - 1}{z - 1} \] ***

    \paragraph{Exemple 3}\label{exemple-3}

La dérivée \(\frac{du}{dt}(kh)\) peut être approximée par:

\[ y(kh) = \frac{u(kh) - u(kh-h)}{h} \]

D'où la fonction de transfert:

\[ G(z) =  \frac{Y(z)}{U(z)} = \frac{1-z^{-1}}{h} \]

Ou encore:

\[ G(z) = \frac{z-1}{hz} \]

Il possède un zéro \(z=1\) et un pôle à l'origine.

Le zéro \(z=1\) est caractéristique de l'effet dérivateur numérique, de
la même manière que le zéro \(s=0\) l'est pour le dérivateur analogique.
***

    \paragraph{Example 4}\label{example-4}

Un retard pur de \(d\) périodes d'échantillonnage s'obtient par:

\[ G(z) = z^{-d} = \frac{1}{z^d} \]

Il possède \(d\) pôles à l'origine. ***

    La fonction de transfert discrète peut prendre la forme suivante:

\[ G(z) = \frac{B(z)}{(z-1)^l A(z)} \]

Elle possède \(l\) pôles \(z=1\), soit \(l\) intégrateurs. L'entier
\(l\) est appelé le type ou la classe du système.

    On obtient le gain permanent du système par:

\[ \gamma = \lim_{z \rightarrow 1} (z-1)^l G(z) = \frac{B(1)}{A(1)} \]

Lorsque \(l\) vaut 0, 1 ou 2, le gain est appelé respectivement gain
statique, en vitesse ou en accélération.

    Enfin, la transformée en \(z\) inverse de \(G(z)\) permet de retrouver
la réponse impulsionnelle \(\left\{g(kh)\right\}\).


    % Add a bibliography block to the postdoc
    
    
    
\end{document}
